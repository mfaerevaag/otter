%%% Local Variables:
%%% mode: latex
%%% TeX-master: "../report"
%%% End:

In this section we will look at how, given our problem
scenario\todo{clearly defined?} and scope of project~\ref{sec:scope},
design an effective and efficient system solving all\todo{all?}
challenges previously mentioned.


\subsection{Topography}

As shown in figure~\ref{fig:topo}, the system is made up of
a central server with an arbitrary number of connected clients.

\begin{figure}[ht]
  \centering
  \includegraphics[scale=0.5]{figures/topography.png}
  \caption{\label{fig:topo} Server - Client topography}
\end{figure}


\subsection{Cryptography}

To solve the issues of confidentiality and integrity we will utilize
cryptography, including both symmetric and asymmetric algorithms.

When two clients connect to the server, as shown in the figure above,
these does not necessarily have any knowledge of each other, yet want
to have full confidentiality in their dialogue. We can solve this with
by exchanging public keys, where both send each other their public
keys. These can then be used by each client to send encrypted messages
to the other.

One could ask if this limits the conversation to only two
participants. Using solely asymmetric cryptography this would be
impractical, but not impossible. Though, by using the public keys to
exchange a symmetric key, an arbitrary number of clients could easily
send and receive encrypted messages with minimal computational
overhead, once the symmetric key had been distributed.

% TLS, MITM
It is important that connections to the server is done through a
secure channel, preferably using TLS encryption. This way, no MITM
attack could compromise the data, or even worse, the key exchanges
done between clients. If an adversary could intercept and change the
key during an exchange, he or she would have the ability to modify or
forge any message in the name of the intercepted party.

% verification
Another benefit of asymmetric, or public-key, encryption is that it
could be used to verify the identify of a client. Instead of
generating an temporary key pair upon connecting to the server, the
client could choose to use an existing key pair. If two clients know
each other they could digital signatures to verify that the other
party. If one does not want to be verified, thus remaining anonymous,
one could simply generate a key-pair which are disposed upon
disconnecting.
