%%% Local Variables:
%%% mode: latex
%%% TeX-master: "../report"
%%% End:

In this section we will discuss how well our solutions handles the
threats presented in our risk analysis, in
section~\ref{tbl:vuln-matrix} and how it compares to other works,
which solve many of the same challenges.


% \subsection{Confidentiality and integrity}
% OTR?
% good shit


\subsection{Anonymity}
% monitor access nodes
% Proxy with tor
% with tor?
% peer to peer?

We have earlier stated that our solution features anonymity, as the
users real identity is never needed by the system. It must be said,
that if an adversary was to monitor the IP addresses of clients
connecting to the central server, their identities could be
compromised. This issue is something that is difficult to prevent, as
the topography of the system is designed to behave is such a way.

A solution to this is to make the application as a Peer-to-Peer (P2P),
as the adversary will not have an obvious place to monitor traffic. We
chose from the start only to discuss this point, but not consider it
during the design or implementation, as it adds considerable
complexity to both.

An alternative solution, which does not change the design of the
system, is to utilize technologies such as TOR~\footnote{TOR -
  \url{https://www.torproject.org/}} which also secures communications
in transit, but also hides the clients origin. The client application
can therefor route its traffic through TOR, using it as a proxy. This
can be done easily with applications such as {\tt
  torify}~\footnote{\url{https://wiki.archlinux.org/index.php/tor\#Torify}}. It
should be mentioned that there are examples where clients have been
identified despite using TOR, for reasons due to bad configuration of
TOR and monitoring of TOR access nodes.


\subsection{Similar works}

There are countless chat protocols and application in existence, but
few which are designed with regards to security. They are in most
cases designed in a time where security was not as big a concern and
has later been patched in some way to try to remedy this.

Open protocols such IRC~\footnote{RFC 2812 -
  \url{https://tools.ietf.org/html/rfc2812}} and XMPP~\footnote{RFC
  3921 - \url{https://xmpp.org/rfcs/rfc3921.html}} are both criticized
for not handling security out-of-the-box~\footnote{IRC Security -
  \url{http://www.irchelp.org/security/}}~\footnote{Securing XMPP -
  \url{https://wiki.xmpp.org/web/Securing_XMPP}}. This has to be
taking into consideration by the user itself, typically through a
secure connection with TLS. Although this secures the transmissions,
the data is still in the clear on the server. Some chat application,
for instance Pidgin which is quite popular~\footnote{Pidgin -
  \url{https://pidgin.im/}}, support third-party add-on software. One
of these add-ons are {\tt pidgin-otr} which adds Off-The-Record (OTR)
messaging. This features full confidentiality and integrity using the
same cryptographic primitives.
