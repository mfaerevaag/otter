%%% Local Variables:
%%% mode: latex
%%% TeX-master: "../report"
%%% End:

When designing mission or safety-critical systems the design, and the
planning there of, is of paramount importance. All challenges which
the product is to solve has to be solved during the design, and not
programmed around during the implementation fase. This would only
create unpredictability and uncertainty regarding to the behavior of
the system. This is if course, not beneficial when attempting to
ensure a system will correctly in a future scenario. In this process
it natural to begin with evaluating which risks are involved.

To evaluate which risks are involved we first need to identify the
systems assets, i.e. what do we want to protect. This could be
intellectual property, documentation, hardware, software or
data~\cite[p. 528]{pfleeger}. For our scenario, we will only be
looking at the latter. In this section we will discuss which types of
data we consider to be an asset, what threats we have to protect
against therefor and what we consider to be the relevant attacker
models. Finally we will define the scope of our project, based on the
results just discussed.


\subsection{Assets}

TODO


\subsection{Vulnerabilities and threats}

TODO


\subsection{Attacker models}

TODO


\subsection{Scope of project}

In this project we will focus on TODO
