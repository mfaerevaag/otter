%%% Local Variables:
%%% mode: latex
%%% TeX-master: "../report"
%%% End:

When creating mission or safety-critical systems, the design, and the
planning there of, is of paramount importance. All challenges which
the product is to solve has to be solved during the design fase, and
not programmed around during the implementation fase. This would only
create unpredictability and uncertainty regarding to the behavior of
the system. This is if course, not beneficial when attempting to
ensure a system will behave correctly in a future scenario. In this
process it natural to begin with evaluating which risks are involved.

To evaluate which risks are involved we first need to identify the
system's assets, i.e. what do we want to protect. This could be
intellectual property, documentation, hardware, software or
data~\cite[p. 528]{pfleeger}. For our scenario, we will only be
looking at the latter, as our system is solely a software product,
independent of any application or purpose.

In this section we will discuss which types of data we consider to be
an asset, what threats we have to protect against therefor and what we
consider to be the relevant attacker models. Finally we will define
the scope of our project, based on the results just discussed.

Throughout this report will use three central concepts regarding
properties of fault-tolerant and security-critical systems, namely
availability, integrity and confidentiality. These properties are what
makes the system at hand valuable to any
user~\cite{anderson}. We will regard \emph{confidentiality}
as the of the message contents, \emph{integrity} as the assurance that
messages cannot be tampered with or forged and \emph{availability} as
the system's ability to be available to the users. Additionally we
introduce \emph{anonymity} as the secrecy of the clients' real
identities.

\subsection{Assets}

As mentioned, our system is fairly simple with regards to assets, as
there is only data, hardware and the software it self that can be
considered as something of value. Data can be represented by both
messages and the surrounding metadata. Metadata that could be valuable
for an adversary are, for instance, identities of clients, such as IP
addresses and names; relationships between clients, i.e. who is
talking to whom; what times are messages being sent, etc. The
integrity and confidentiality of the system are key qualities.

Software includes the system itself, but also the eventual products
used on web-servers, operating systems etc.

Hardware includes the server(s) which the system runs on.


\subsection{Vulnerabilities and threats}

We will continue with identifying the vulnerabilities threatening our
system. In order to do this we use the C-I-A triad~\cite{anderson},
which introduces four acts that could compromise the systems assets or
otherwise cause harm, namely; interception, interruption, modification
and fabrication. All can be viewed as a breach of earlier mentioned
qualities of system, where interception is as a breach of
confidentiality, interruption of availability, and modification and
fabrication of integrity. We will briefly discuss each in turn, with
regards to the data.

Confidentiality ensures that data only can be read by authorized
parties. If an adversary is passively listening in to communications,
these should not reveal any information. An obvious solution is the
use of cryptography to do encryption before transmission.

Availability can be compromised by denying a user of using the
system. This can be done with Denial of Service (DoS) attacks, which
are hard to defend against. A DoS attack is essentially the act of
creating large amounts of traffic to the servers providing the service
to deny other users service. As there is hard for a server to
distinguish legitimate traffic and not, which is why it is a common
vulnerability for most public systems. Another attack would be
compromising the server which the service runs on. Both of these
attacks is not something detected by our system and will therefore be
regarded as out of the scope of this project.

Integrity is arguably one of the most important traits with a
security-critical system. Even worse than denying service to a user is
providing him or her with fraudulent information. If an adversary was
able to do a Man in the Middle (MITM) attack, where he or she does not
only observe the traffic, but actively alter it, the adversary can
modify or create fraudulent data from an arbitrary user to another.

\begin{table}[h!]
  \centering
  \begin{tabular}{l|l|l|l}
    \multicolumn{1}{c|}{} & \multicolumn{1}{c|}{\textbf{Confidentiality}} & \multicolumn{1}{c|}{\textbf{Integrity}} & \multicolumn{1}{c}{\textbf{Availability}} \\ \hline
    Intellectual Property &  & & \\ \hline
    Documentation & & & \\ \hline
    Hardware & & backdoor & stolen, destroyed \\ \hline
    Software & & backdoor & deleted, license expire \\ \hline
    Data & disclosure, inference & tampered & deleted, ransom ware
  \end{tabular}
  \caption{\label{tbl:vuln-matrix} Vulnerability Matrix}
\end{table}

Lastly, the software and hardware is vulnerable to being stolen or
removed, infected with malware or by licenses and such expiring and
therefore denying service. To summarize, figure~\ref{tbl:vuln-matrix}
shows the different threats to our system, though only considering
data.


\subsection{Attacker models}

Likely attackers of our system can be split into different
categories, depending on their motives for attack. Again motives will
differ depending on the users of the system.

The first and least harmful category would be private people our small
groups, having a personal incentive for either denying service or
reading someones messages. We do not regard these as a significant
threat as they do not posses the financial backing or knowledge to
break an a system using modern cryptographic standards.

If it was a corporation using the system obvious attackers would be
competing companies. These would likely have finical incentives for
attacking and therefore be greatly financed, causing a great threat to
availability through DoS attacks, for instance.

Assuming the system is available to everyone, and not a specific group
of people, we could take case of the PRISM program in the US, where
governmental institution with enormous financial power, would like to
monitor our system. This would arguably be the systems greatest
adversary as their attack is both hard to detect and can compromise
confidentiality, or even integrity.



\subsection{Scope of project}
\label{sec:scope}

In this project we will focus on ensuring confidentiality and
integrity is upheld, regardless of attacker. Users will always remain
anonymous by choice and not have any data or metadata disclosed
without intentionally doing so.

TODO mention anonymity, accountability, verification, etc?
