%%% Local Variables:
%%% mode: latex
%%% TeX-master: "../report"
%%% End:

\subsubsection{motivation}
We wanted to explore the concept of private and
confidential communication as we feel that commonly used messaging
solutions do not provide enough. We aim to do so using RSA as our
encryption standard and using TLS as encryption for the connections.
...
...
...


scenario
create a scenario or just more motivation??
discuss:

\subsubsection{security analysis}
There are many ways to attack a system depending on what you want and
what your attacking. We aim to have confidentiality and privacy by
securing our `application'~\todo{get better word} from at least the
following attacks:
\begin{itemize}
\item Man in the middle(MtiM)
\item Chosen-ciphertext attack (CCA)
\item Ciphertext-only attack (COA)
\item Time/memory tradeoff attacks (TMTA)
\end{itemize}

\paragraph{Confidentiality}
How sure can we be that the message we send or receive have not been tamperd with or has been already read

- privacy (anon, auth)

\paragraph{Privacy}
\paragraph{Anonymity}
is a cornerstone of the internet, most comments are done
anonymously using unidentifiable pseudonyms. Though these usernames
can be used to identify the user they have the possibility of being
separated and anonymous from the actor. Anonymity on the internet is
basically that a comment you state will not be able to be tracked back
to you.

A big problem for anonymity is IP adresses as they serve as an virtual
address. These can be mapped to your Internet Service Provider (ISP),
who can provide customer information as they know what IP addresses
are leased to whom. This does not implicate a specific indivudual but
provides regional information and is powerful circumstantial
evidence.\todo{good bad sides of anon?}



\paragraph{Authentication}






%% The authentication of information can pose special problems with
%% electronic communication, such as vulnerability to man-in-the-middle
%% attacks, whereby a third party taps into the communication stream, and
%% poses as each of the two other communicating parties, in order to
%% intercept information from each. Extra identity factors can be
%% required to authenticate each party's identity.

%% The term digital authentication refers to a group of processes where
%% the confidence for user identities is established and presented via
%% electronic methods to an information system. It is also referred to as
%% e-authentication. The digital authentication process creates technical
%% challenges because of the need to authenticate individuals or entities
%% remotely over a network. The American National Institute of Standards
%% and Technology (NIST) has created a generic model for digital
%% authentication that describes the processes that are used to
%% accomplish secure authentication:

%% Enrollment – an individual applies to a credential service provider
%% (CSP) to initiate the enrollment process. After successfully proving
%% the applicant’s identity, the CSP allows the applicant to become a
%% subscriber.  Authentication – After becoming a subscriber, the user
%% receives an authenticator e.g., a token and credentials, such as a
%% user name. He or she is then permitted to perform online transactions
%% within an authenticated session with a relying party, where they must
%% provide proof that he or she possesses one or more authenticators.
%% Life-Cycle maintenance – the CSP is charged with the task of
%% maintaining the user’s credential of the course of its lifetime, while
%% the subscriber is responsible for maintaining his or her
%% authenticator(s).[1][8]
