%%% Local Variables:
%%% mode: latex
%%% TeX-master: "../report"
%%% End:

We have provided together with this report, a Proof of Concept (PoC)
application, featuring a server and client application. Due to time
limitations we have not implemented all features described in this
report, but rather laid the ground work proving its feasibility.

In this section we will describe how each component of the system is
implemented with regards to the design, and what choices were made
therefor. We will state when a feature is not implemented in our PoC,
but also explain how we have planned to implement it based on our
already implemented design.


\subsection{Rust}
We decided to implement our project a low-level language as the
development is challenging and fulfilling. The many pitfalls in c/c++,
primarily memory safety can lead to many serious bugs such as buffer
overflows, memory leaks, double deallocations, and dangling pointers.

So we wanted a language where the memory management/Garbage collection
was handled automatically. Rust does this in a very pragmatic way by
giving binding a value to exactly one place and one place
only. Because of this the resources that the variables hold (memory,
file handles, sockets, etc.) can be automatically freed when the
variable goes out of a scope (defined by code blocks within curly
braces, \{ and \}).


\subsection{Server and Client}
As mentioned, we created PoC terminal application that uses a server
and an arbitrary number of clients. Using the Rust package system,
{\tt cargo}, we can include several binary target in the same
project. The implementation includes the follwoing files, split into
server and client binaries:
\begin{figure}[h]
  \begin{minipage}{0.5\textwidth}
    \paragraph{Server}
    \begin{itemize}
      \item main.rs
      \item socket.rs
      \item engine.rs
      \item command.rs
      \item state.rs
    \end{itemize}
  \end{minipage} \hfill
  \begin{minipage}{0.5\textwidth}
    \paragraph{Client}
    \begin{itemize}
      \item main.rs
      \item socket.rs
      \item engine.rs
      \item utils.rs
    \end{itemize}
  \end{minipage} \vfill
  \caption{Server and client}
\end{figure}

\subsubsection{Server}
For our service to function properly we need a server to recieve data
and to pass it along to the inteded client. This is achived through
network sockets, as explained in the following section. They are
simple enough to use but due to our intent of having $'N'$ users when
$N > cores$ we will run in to blocking. This is handled by the by
creating an event loop that, creates and runs a web socket in its own
thread. As an event occurs, for instance an incoming message, an event
handler is triggered.

The {\tt main}-function is the entry point of every Rust application,
as in many other languages. The {\tt main.rs} starts an instance of
the engine, defined in {\tt engine.rs}, the address and port to
connect to the server is set here. {\tt engine.rs} starts listens to a
specific port (hardcoded to 10000) waiting for incoming sockets. The
task of keeping track of all connected sockets is delegated to {\tt
  state.rs}. It holds a dynamic key-value collection, a hash set, with
the nick of the client, which is generated on connection, as key and
socket as value. As all sockets would like to read from this
collection from their respective threads, it is protected by a
semaphore-like value, called a {\tt RwLock}, ensuring that only can do
this at a time.

When a socket is triggered with a message, it is given to the engine
so it can decide how it should be handled. As explained in
section~\ref{sec:design-ui}, there is a command system. We have
implemented this by parsing the message into a structure represented
by a command and its arguments. For instance, if the command is of
type {\tt /msg} it will send the message forward to its intended
receiver. If the message does not follow the required format, an error
is thrown, defined in {\tt error.rs}, and passed back to the socket
which informs its counterpart accordingly.

\subsubsection{Client}
All our encryption happens client-side in our main function we either generate
a set of keys, or we load a key from file which can be used a identity
verification.

The client implementation is very simple. There are two threads; one
waiting to human input to send, and one waiting for data from the
server. When sending or receiving, the data is either encrypted or
decrypted accordingly.


\subsection{Communication Protocol}
When choosing the protocol which the server and clients will
communicate through, we chose to use web sockets. A socket provides a
convenient handle which the two end points can conveniently pass data
through. To relieve ourself of the major work load of making a web
socket library in Rust, we chose an existing open source
implementation named {\tt
  ws-rs}~\footnote{\url{https://github.com/housleyjk/ws-rs}}.

The encoding of the data sent over this link are simply chosen as
UTF8, as all Rust strings are already UTF8 encoded. As we found, this
offers a challenge as the cipher texts that RSA produces are not UTF8,
and are therefore not safe to parse directly as a UTF8 string. To
solve this we first encode the cipher text to base64, which only uses
UTF8 safe characters, and there after send as a string. The receiver
would simply decode the base64 to bytes and decrypt.


\subsection{Cryptography}
Currently our clients loads a private and public key which they use
for encryption and decryption. We did not get the time to properly
share the public keys with the server, to the clients can exchange
keys, as explained in the designed. To achieve this we would simply
add a field to the socket structure, in {\tt socket.rs}, and setup a
synchronization process when a new sockets connection. The new socket
would be sent the public key of its counterpart by adding a new
command in {\tt command.rs}, for instance called {\tt /keyshare}.

Also, the implementation includes chat rooms, which is essentially a
list of sockets. We did not have time to implement such that clients
to joins the room and share a symmetric key for convenient
encryption. Though, this would be done much in the same manner as
above, where the we would add a field to the room structure, which
held a symmetric key that could be distributed to the
clients. Alternatively the clients could distribute the keys to the
new clients them selves, keeping the server blind to the content to
the message which are exchanged. To do this the client would simply
encrypt the symmetric key using the recipients public key and send
it. To make it so that the receiver can verify the origin of the
message, the client could encrypt the key with its own private key,
essentially signing the key, and then encrypt it with the receivers
public key and send.


\subsection{Requirements}
Now we will go through the main libraries and explain what they do for
us. As mentioned, {\tt ws-rs} provides us with a implementation of
web sockets using
the {\tt MIO} library.

The {\tt openssl} library, or ``crate'' in Rust-speak, is the
toolkit for the TLS protocol.

Lastly, {\tt rustc-serialize} proved a library of base64 encoding the
cipher texts, as explained in above.

% TODO dont think we need to mention these
% {\tt bitflags}  {\tt idna}, {\tt miom},
% {\tt miow}, {\tt net2}, {\tt sha1}, {\tt slab}, {\tt url}.

%%% Local Variables:
%%% mode: latex
%%% TeX-master: "../report"
%%% End:
